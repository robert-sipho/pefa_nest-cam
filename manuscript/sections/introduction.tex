\section{Introduction}

% pointers from cahill
% if a point isn't directly needed to set up the research questions, cut it
% introductions should not be long
% End with a very clear set of specific research questions. Take a long time to really think about how these are worded and what order you want to present them in.

% Paper thesis 
% tradeoffs between brooding and provisioning
%   - spending time brooding is time that can't be invested in provisioning
%   - inclement weather necessitates brooding, but may also increase needs for provisioning due to thermal regulation
%   - inclement weather in the form of rainfall may also reduce hunting ability
% investigating these tradeoffs relies on near continuous monitoring, something that is only achieved through passive technology
% Although such technology 




% Brooding parental behaviour literature
% \cite{chastel2002I}
% \cite{wolf1990AB}
% \cite{santos2012JEB}
% \cite{matysiokova2014FiZ}










%~~~~~~~~~~~~~~~~~~~~~~~~~~~~~~~~~~~~~~~~~~~~~~~~~~~~~~~~~
% importance of nest observation data and associated disturbance
Nest monitoring is an important component of avian population studies.
It enables the assessment of factors that influence breeding success, and is essential to inform conservation and management decisions.
The collection of this data, however, comes at a cost.
Frequently visiting a nest causes disturbance that may lead to reduced nest attendance by parents (), increased conspicuous parental behaviour (), increased probability of predation (), and increased stress in both parents and nestlings ().
The resulting effects of nest visits have the potential to bias results, and impact reproduction.

 
 
 
 
%~~~~~~~~~~~~~~~~~~~~~~~~~~~~~~~~~~~~~~~~~~~~~~~~~~~~~~~~~
% Passive monitoring to reduce cost
Many methods of passive nest monitoring have been introduced to reduce the impact.
One method is to place temperature sensors in the nest to track incubation and brooding.
Others have used radio frequency identification tecnologies to monitor nest usage \cite{alba2019ZB}.
A more common method involves photo and video technology.
Resulting from technological advances in image sensors and data storage, off-the-shelf motion sensitive cameras have become more discrete, and capable of capturing increasingly higher quality images.
Camera settings can be tailored to the needs of a monitoring program with varying options of motion sensitivity, rapid image capturing upon motion triggers, and complete freedom over the duration of time between timelapse images. 
Such capabilities result in much higher resolution data due to near real-time surveilance of avian nesting biology, and a substantial reduction in disturbance caused by nest visits.
Common applications of this technology include the investigation of provisioning behaviour, nest predator identification \cite{cox2012LE,degregorio2014EE}.
\todo{testing todo here}

 
 
 
 
%~~~~~~~~~~~~~~~~~~~~~~~~~~~~~~~~~~~~~~~~~~~~~~~~~~~~~~~~~
% Cost of data processing
The benefits of using nest cameras are substantial, but only after considerable effort has been invested into the processing of images.
Converting images captured at the nest into useable data often requires the manual sorting and classification of images.
Not only is this process time-consuming and error-prone, it often has to be repeated if new research questions are proposed.
Combined with the amount of data collected by cameras each breeding season, processing times bottleneck the researcher's ability to answer ecological questions in a timely manner.

 
 
 
 
%~~~~~~~~~~~~~~~~~~~~~~~~~~~~~~~~~~~~~~~~~~~~~~~~~~~~~~~~~
% Machine Learning
The bottleneck of processing images captured by nest cameras is part of a growing trend in Ecology; the amount of tools available to help collect data is growing faster than our ability to process the data. 
The need for additional tools that automate this process has been identified recently

reducing nest visit frequency while maintaining data quality have been proposed, and the utility depends Methods of passive nest monitoring that reduce the need for frequent visits have been outlined


yet there is an important tradeoff between gathering data on nesting biology and minimizing the disturbance induced by nest visits.

Nest site observations are important for monitoring programs
* provisioning rates
* behavioural responses to inclement weather
* egg-laying
* nestling survival
* band recovery


 

 
 






