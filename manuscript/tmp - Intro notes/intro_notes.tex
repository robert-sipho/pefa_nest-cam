\documentclass[]{scrartcl}

%opening
\title{Camera Trap MS Intro Notes}
\author{Erik Hedlin}

\begin{document}

\maketitle

\section*{Introduction Outline}
\begin{itemize}
	\item Describe the breeding data logged with camera traps
	\begin{itemize}
		\item adult attendance
		\item number of eggs
		\item number of nestlings
		\item lay date
		\item hatch date
		\item band ID
	\end{itemize}
	\item explain the advantages of using cameras
	\begin{itemize}
		\item near real time
		\item continuous, passive monitoring
	\end{itemize}
	\item talk about the disadvantages
	\begin{itemize}
		\item cumbersome to go through thousands of images
	\end{itemize}
	\item introduce the conecpt of machine learning
	\begin{itemize}
		\item offer a mini review of other studies that have used it in the context of ecology
	\end{itemize}
\end{itemize}

\newpage

\subsection*{Breeding data and its importance for population dynamics}
Determining and quantifying factors that drive population dynamics is a central goal of ecology and wildlife management. 

Annual variation in breeding productivity is strongly linked to population dynamics, and is therefore an important area of focus for studies aiming to 
quantify factors affecting population dynamics.

For birds, this includes monitoring factors related to breeding productivity such as breeding phenology, clutch size, parental behaviour, nestling survival, and adult survival.

Collecting data that accurately describes such factors can be difficult however.


\begin{env}
	
\end{env}

\newpage

\subsection*{Advantages of using cameras}


\newpage

\subsection*{Disadvantages of using cameras}
\newpage

\subsection*{Review of machine learning}
\end{document}
