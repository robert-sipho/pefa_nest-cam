\documentclass[preprint,review,12pt]{elsarticle}
\usepackage{amssymb}
\usepackage{lineno}
\journal{Journal Name}

% Colors %%%%%%%%%%%%%%%%%%%%%%%%%%%%%%%%%%%%%%%%%%%%%%%%%%%%%%%%%%%%%%%%%%%%%
\usepackage{xcolor}
\definecolor{red1}{HTML}{620012} 
\definecolor{red2}{HTML}{AD1520} 
\definecolor{red3}{HTML}{C14D51} 
\definecolor{red4}{HTML}{DA9193} 
\definecolor{red5}{HTML}{F7DFE0} 

\definecolor{blue1}{HTML}{004366} 
\definecolor{blue2}{HTML}{095E89} 
\definecolor{blue3}{HTML}{668092} 
\definecolor{blue4}{HTML}{AFC0C9} 
\definecolor{blue5}{HTML}{D0E0E8} 

\definecolor{grey1}{HTML}{191919} 
\definecolor{grey2}{HTML}{424242} 
\definecolor{grey3}{HTML}{5E5E5E} 
\definecolor{grey4}{HTML}{7F7F7F} 
\definecolor{grey5}{HTML}{9C9C9C} 
\definecolor{grey6}{HTML}{CCCCCC} 
\definecolor{grey7}{HTML}{E0E0E0} 
\definecolor{grey8}{HTML}{EFEFEF} 
\definecolor{grey9}{HTML}{F4F4F4}


\definecolor{yellow1}{HTML}{ffa92d}
\definecolor{yellow2}{HTML}{ffdcaa}



% Fonts  %%%%%%%%%%%%%%%%%%%%%%%%%%%%%%%%%%%%%%%%%%%%%%%%%%%%%%%%%%%%%%%%%%%%%%
% Language and font encodings
\setlength{\marginparwidth}{2cm}
\usepackage[color = yellow2, textwidth=4cm, colorinlistoftodos, size = scriptsize, linecolor=yellow2, bordercolor=yellow2]{todonotes}
\usepackage{graphicx}


\usepackage{hyperref}
\hypersetup{ 
	colorlinks = true,
	citecolor  = red1,
	urlcolor   = red1,
	menucolor  = black,
	linkcolor  = red1} 

\usepackage[english]{babel}
\usepackage[utf8x]{inputenc}
\usepackage{cochineal}
\usepackage[OT1,LGR,T2A,T1]{fontenc}
\usepackage[cochineal,varg]{newtxmath}
% \usepackage[numbers, square, comma]{natbib}
%\usepackage{roboto}  %% Option 'sfdefault' only if the base font of the document is to be sans serif

%\usepackage[defaultsans,oldstyle]{opensans} %% Alternatively
%% use the option 'defaultsans' instead of 'default' to replace the
%% sans serif font only.

%%%%%%%%%%%%%%%%%%%%%%%%%%%%%%%%%%%%%%%%%%%%%%%%%%%%%%%%%%%%%%%%%%%%%%%%%%%%%%%


\begin{document}
\listoftodos
\begin{frontmatter}
%
% paper title
% Titles are generally capitalized excepts for words such as a, an, and, as,
% at, but, by, for, in, nor, of, on, or, the, to and up, which are usually
% not capitalized unless they are the first or last word of the title.
% Linebreaks \\ can be used within to get better formatting as desired.
% Do not put math or special symbols in the title.
\title{Deep Learning Reveals Links Between Brooding Behaviour and Inclement Weather}



% make the title area
\maketitle

% not sure who will be first author yet, Robert and Erik to discuss
\author{Robert Winter, Erik Hedlin, Alastair Franke}
\address{Edmonton, Alberta}



\begin{abstract}

% importance of fine scale breeding data

% difficulty of obtaining fine scale breeding data

% difficult of processing fine scale breeding data

% use of machine learning in our project

% results obtained from machine learning.
Motion sensor camera traps are used extensively in the fields of zoology and ecology. Camera traps have been around for decades and have revolutionized wildlife research and conservation due to their ability to capture information with little expense, and minimal disturbance to the wildlife. Recently, the ability of computers to recognize certain aspects of images has led to the use of these techniques to save human time in examining the camera trap images. In this work object detection methods are applied to peregrine nest cameras. Our application of these techniques is able to achieve XX\% accuracy. Additionally due to the nature of the sequence of images taken, any mistakes made can be rectified by using a time-sequence of the data. 

\end{abstract}


\begin{keyword}
Machine Learning \sep Brood Rearing \sep Climate Change
\end{keyword}

\end{frontmatter}

 % start line numbering
\linenumbers


% chapters

\section{Introduction}

% pointers from cahill
% if a point isn't directly needed to set up the research questions, cut it
% introductions should not be long
% End with a very clear set of specific research questions. Take a long time to really think about how these are worded and what order you want to present them in.

% Paper thesis 
% tradeoffs between brooding and provisioning
%   - spending time brooding is time that can't be invested in provisioning
%   - inclement weather necessitates brooding, but may also increase needs for provisioning due to thermal regulation
%   - inclement weather in the form of rainfall may also reduce hunting ability
% investigating these tradeoffs relies on near continuous monitoring, something that is only achieved through passive technology
% Although such technology 


% Brooding parental behaviour literature
% \cite{chastel2002I}
% \cite{wolf1990AB}
% \cite{santos2012JEB}
% \cite{matysiokova2014FiZ}










%~~~~~~~~~~~~~~~~~~~~~~~~~~~~~~~~~~~~~~~~~~~~~~~~~~~~~~~~~
% importance of nest observation data and associated disturbance
Nest monitoring is an important component of avian population studies.
It enables the assessment of factors that influence breeding success, and is essential to inform conservation and management decisions.
The collection of this data, however, comes at a cost.
Frequently visiting a nest causes disturbance that may lead to reduced nest attendance by parents (), increased conspicuous parental behaviour (), increased probability of predation (), and increased stress in both parents and nestlings ().
The resulting effects of nest visits have the potential to bias results, and impact reproduction.

 
 
 
 
%~~~~~~~~~~~~~~~~~~~~~~~~~~~~~~~~~~~~~~~~~~~~~~~~~~~~~~~~~
% Passive monitoring to reduce cost
Many methods of passive nest monitoring have been introduced to reduce the impact.
One method is to place temperature sensors in the nest to track incubation and brooding.
Others have used radio frequency identification tecnologies to monitor nest usage \cite{alba2019ZB}.
A more common method involves photo and video technology.
Resulting from technological advances in image sensors and data storage, off-the-shelf motion sensitive cameras have become more discrete, and capable of capturing increasingly higher quality images.
Camera settings can be tailored to the needs of a monitoring program with varying options of motion sensitivity, rapid image capturing upon motion triggers, and complete freedom over the duration of time between timelapse images. 
Such capabilities result in much higher resolution data due to near real-time surveilance of avian nesting biology, and a substantial reduction in disturbance caused by nest visits.
Common applications of this technology include the investigation of provisioning behaviour, nest predator identification \cite{cox2012LE,degregorio2014EE}

 
 
 
 
%~~~~~~~~~~~~~~~~~~~~~~~~~~~~~~~~~~~~~~~~~~~~~~~~~~~~~~~~~
% Cost of data processing
The benefits of using nest cameras are substantial, but only after considerable effort has been invested into the processing of images.
Converting images captured at the nest into useable data often requires the manual sorting and classification of images.
Not only is this process time-consuming and error-prone, it often has to be repeated if new research questions are proposed.
Combined with the amount of data collected by cameras each breeding season, processing times bottleneck the researcher's ability to answer ecological questions in a timely manner.

 
 
 
 
%~~~~~~~~~~~~~~~~~~~~~~~~~~~~~~~~~~~~~~~~~~~~~~~~~~~~~~~~~
% Machine Learning
The bottleneck of processing images captured by nest cameras is part of a growing trend in Ecology; the amount of tools available to help collect data is growing faster than our ability to process the data. 
The need for additional tools that automate this process has been identified recently

reducing nest visit frequency while maintaining data quality have been proposed, and the utility depends Methods of passive nest monitoring that reduce the need for frequent visits have been outlined


yet there is an important tradeoff between gathering data on nesting biology and minimizing the disturbance induced by nest visits.

Nest site observations are important for monitoring programs
* provisioning rates
* behavioural responses to inclement weather
* egg-laying
* nestling survival
* band recovery


 

 
 







\section{Methods}
% NOTES FROM CAHILL
% Relate everything you talk about to the research qustions described above
% Do not swap the order, such that if you list questions 1, 2, 3 in the intro, do not discuss the methods as 3, 2, 1
% When you discuss stat methods, be sure to relate each test to a specific research objective. If the test is complciated, let the reader know what type of stat result would indicate what type of answer to your question.
 
 
 
 
%~~~~~~~~~~~~~~~~~~~~~~~~~~~~~~~~~~~~~~~~~~~~~~~~~~~~~~~~~
\subsection{Study Area}

The study area is located on the western coast of Hudson Bay, and encompasses a 422 km\textsuperscript{2} area that surrounds the community of Rankin Inlet, Nunavut, Canada (62°49'N, 92°05'W)(Figure 1). 
The terrestrial portion is characterized by rolling mesic tundra interspersed with numerous lakes and streams, and supports communities of passerines, shorebirds, ducks, geese, and small mammals. 
The marine portion is composed of numerous islands of varying size and also supports diverse bird communities, in addition to small mammals. 
Rocky outcrops that form cliffs are common throughout both terrestrial and marine areas, and provide ideal nesting habitat for raptor species such as Rough Legged Hawks (\textit{Buteo lagopus}), Peregrine Falcons (\textit{Falco peregrinus}), Common Ravens (\textit{Corvus corax}) occasionally Golden Eagles (\textit{Aquila chrysaetos}), and Gyrfalcons (\textit{Falco rusticolus}).   
 
 
 
%~~~~~~~~~~~~~~~~~~~~~~~~~~~~~~~~~~~~~~~~~~~~~~~~~~~~~~~~~
\subsection{Peregrine Monitoring}

Peregrine Falcon population monitoring was conducted over five breeding seasons from 2013 to 2017.
Site occupancy surveys began in May as peregrines arrived on site from migration.
Regardless of whether or not we detected breeding pairs, all suitable territories were monitored until the season had advanced sufficiently into the incubation period and we could confidently conclude that vacant sites would remain vacant.


 
 
 
%~~~~~~~~~~~~~~~~~~~~~~~~~~~~~~~~~~~~~~~~~~~~~~~~~~~~~~~~~
\subsection{Camera Work}

To capture lay dates, hatch dates, causes of nestling mortality, and parental behaviour throughout the breeding season, motion sensitive cameras (RECONYX, Holmen Wisconsin, USA, models PC85 and PC800, 2013 n=11, 2014 n=22, 2015 n=22, 2016, 2017)  %%%%%%%% FILL IN YEARS %%%%%%%%%%
were placed 60 - 200 cm from the nest bowl. 
Although we aimed to visit nests every 5 days to replace camera batteries, many of the nesting territories are difficult to access during inclement weather. 
As such, camera settings such as motion sensitivity and the duration of time between timelapse images, were tailored to each site to preserve battery power in the event that we could not return in five days.
Typical camera settings for accessible nest sites included timelapse images every 15 minutes regardless of motion, three rapid pictures when the camera was triggered by motion, followed by a quiet period of 5 seconds within which the camera did not respond to movement, while typical camera settings for difficult to access sites included only one image captured when motion was detected, and a duration of 5 minutes between time-lapse images.
Although the camera settings resulted in fewer captured images for difficult to access sites, the images were more evenly spaced and effectively surveyed events of interest at the nest site.


%~~~~~~~~~~~~~~~~~~~~~~~~~~~~~~~~~~~~~~~~~~~~~~~~~~~~~~~~~
\subsection{Transfer Learning}

 Data outputed by the CNN is binary, with a 1 provided when a certain class of object is detected in an image.

 
 
%~~~~~~~~~~~~~~~~~~~~~~~~~~~~~~~~~~~~~~~~~~~~~~~~~~~~~~~~~
\subsection{Modeling}

 

%~~~~~~~~~~~~~~~~~~~~~~~~~~~~~~~~~~~~~~~~~~~~~~~~~~~~~~~~~
% Convert to proportion
To investigate nest attendance among breeding pairs of peregrine falcons, we first converted the presence absence data outputed by the Convolutional neural networks into the proportion of each day adults were at the scrape.
Depending on the extent of motion, the images captured by remote cameras are often irregular.
Converting irregular series of images into a total proportion of time spent at the nest was completed by first generating an empty dataframe that contained a row for each minute within each day. 
The cell correpsonding to a nest site in a given minute was filled with a `1' if the CNN detected an adult at any time within that minute.
If no images were captured within that minute, we assumed the value of the previous minute in which an image was captured still applied.
If no images had been captured during the amount of time correpsonding to timelapse intervals (15m), no value was carried forward.
This method assumes that the cameras reliably detected occupancy state changes at a nest, or in other words, cameras reliably captured the arrival and departure adults.
Since arrivals and departures cause substantial motion (going from perched to flight), we are confident this is a safe assumption.

Nestlings explore the cliff surrounding the scrape as they grow, and at a certain point, we can no longer be certain that cameras are reliably capturing nest attedance.
% FILL <- 
To protect against possible bias oirignating from nestling movement, we truncated our data at a total brood age of 15 days.
 

%~~~~~~~~~~~~~~~~~~~~~~~~~~~~~~~~~~~~~~~~~~~~~~~~~~~~~~~~~
% Model

$$
\begin{align}
y_i &\sim Beta(a_i, b_i) \\
%\mu_i &= \frac{a_i}{a_i + b_i} \\
a_i &= \mu_i * \theta \\
b_i &= (1 - \mu_i) * \theta \\
logit(\mu_i) &= \alpha + \beta*X_i
\end{align}
$$

To model the daily proportion of time adults attended nests, we programmed a generalized linear mixed model with a beta-binomial probability distribution and logit link function in R using R2jags \cite{}. 
% FILL <- 
Daily proportions are potentially autocorrelated, so we investigated various residual varince structures to estimate correlation between timesteps.
Data was repeat measures, so we included nested random intercepts with brood nested in year.
We suspected that adult nest attendance is related to yearly conditions, and therefore estimated random slopes for year.
Lastly, we included various combinations of fixed effects that we suspected were important in explaining nest attendance. 
All priors were specified as diffuse, and we employed a workflow that followed \cite{} % bayesian workflow, gelman et al.


%~~~~~~~~~~~~~~~~~~~~~~~~~~~~~~~~~~~~~~~~~~~~~~~~~~~~~~~~~
% fixed effects

% weather
Daily climate data included in the models were obtained from an Environment Canada weather station positioned in a central location in the study area. 

% nestling data
All nestling data was gathered every 5 days as part of routine nest visits, and this included nestling weights, brood size, and brood age (as determined by the age of the oldest nestling).


\section{Results}
% Notes from cahill
% For nearly every paper, this should be your shortest section. For a regular, ~20 page paper, the results of a tightly written paper with a strong story should be about 1 page (excluding tables/figures)
% use fewer figures and tables than you think you need. put the extras online. The problem with figures is that simple ones are more briefly stated with text, while complex ones take a long time to understand. The latter is fine if, and only if, they are directly related to your main research questions. If tables or figures are simply supportive, then putting them in the main paper will greatly subtract from your overall pitch.
% When discussing statistical results, focus on the answer to your research questions, not test statistics, P values, or AIC values. These are tools for interpretation, they are not meaninful in and of themselves. They are to be used to support your story
% answer your research questions in the same order you presented them

\section{Discussion}
% Notes from Cahill
% Discuss your research questions in the same order you originally presented them.
% when interpreting, it is essential that you come back to the same ideas you laid out in your introduction, but now indicate how your results alter our understanding. If some ideas in your introduction don't get referred to in the discussion, they proably didn't belong in your intro
% You should extrapolate from your results one step, but no more than that. For example, if you found X, you can suggest Y. But you cannot say that since X is true, Y might be too, and therefore Z happens
\section{Results}

 
 
 
 
%~~~~~~~~~~~~~~~~~~~~~~~~~~~~~~~~~~~~~~~~~~~~~~~~~~~~~~~~~
% Methods

 
 
 
 
%~~~~~~~~~~~~~~~~~~~~~~~~~~~~~~~~~~~~~~~~~~~~~~~~~~~~~~~~~
% Results

 
 
 
 
%~~~~~~~~~~~~~~~~~~~~~~~~~~~~~~~~~~~~~~~~~~~~~~~~~~~~~~~~~
% Discussion





%
\bibliography{mybib}
\bibliographystyle{plainnat}
%


\end{document}


